\documentclass{article}
\usepackage{graphicx}
\usepackage{amsmath, amssymb}
\usepackage{hyperref}
\usepackage{listings}

\title{\textbf{Assignment: Identifying Critical Nodes in a Computer Network Using DFS}}
\author{Course: Algorithms: Design \& Analysis}
\date{\today}

\begin{document}
\maketitle

\section{Problem Statement}
A critical node (articulation point) in a computer network is a node whose removal increases the number of connected components. Such nodes are crucial in network resilience analysis. In this assignment, you will:

\begin{enumerate}
    \item Model a computer network as an undirected graph, where nodes represent routers/servers, and edges represent connections.
    \item Implement an algorithm to detect articulation points using Depth-First Search (DFS).
    \item Simulate network failure by removing a critical node and analyzing the resulting connectivity.
\end{enumerate}

\section{Input Format}
The input file (\texttt{network\_<file number>.txt}) will contain:
\begin{itemize}
    \item An integer \(V\) (number of nodes) and \(E\) (number of edges).
    \item \(E\) lines, each containing two integers \(u, v\) indicating an undirected edge between \(u\) and \(v\).
\end{itemize}

\textbf{Example Input:}
\begin{verbatim}
6 7
0 1
0 2
1 3
2 3
3 4
3 5
4 5
\end{verbatim}

\section{Output Format}
\begin{itemize}
    \item A list of articulation points (if any).
    \item A network failure simulation in which one articulation point is removed, showing the resulting disconnected components.
\end{itemize}

\textbf{Example Output:}
\begin{verbatim}
Critical Nodes (Articulation Points): 3
Removing node 3 will disconnect the network into two components: [0, 1, 2] and [4, 5]
\end{verbatim}
\section{Algorithm Requirements}
Your implementation should:
\begin{itemize}
    %\item Implement Tarjan's Algorithm to find articulation points.
    \item Use DFS traversal to compute the discovery time (disc []) and the lowest reachable ancestor (low[]). So we maintain an additional array low[] such that:
low[u] = min(disc[u], disc[w]) , Here w is an ancestor of u and there is a back edge from some descendant of u to w.
    \item Identify nodes that satisfy the conditions of the articulation point: \begin{itemize}
            \item A root with two or more children.
            \item A non-root node where low[child]$\ge$disc[node].
          \end{itemize}
    \item Simulate network failure by removing an articulation point and computing new connected components.
\end{itemize}

\section{Implementation Details}
Use \textbf{Python} for implementation. Your program should:

\begin{itemize}
    \item Read input from files named \texttt{network\_<file number>.txt} located in the \textbf{network} folder (\url{https://tinyurl.com/2p9n3hvb}).
    \item Compute \textbf{articulation points} using \textbf{Depth-First Search (DFS)}.
    \item Simulate and output the effect of \textbf{network failures} caused by removing articulation points.
\end{itemize}

Ensure that your program dynamically processes all files in the \textbf{network} directory, handling multiple input files systematically.

\section{Evaluation Criteria}
Your assignment will be graded based on:
\begin{itemize}
    \item \textbf{Correctness (50\%)}: Identifies articulation points accurately using DFS.
    \item \textbf{Simulation (30\%)}: Visualize the effect of \textbf{network failures} caused by removing articulation points.
    \item \textbf{Code Readability (20\%)}: Well-commented and structured code.
\end{itemize}

\section{Submission Guidelines}
\begin{itemize}
    \item Submit your Python script as \texttt{network\_analysis\_<Your name>.py}.
    \item Provide a short report (\texttt{report.pdf}) covering:
          \begin{itemize}
            \item Your approach to solving the problem.
            \item Challenges faced and resolutions.
            \item Example output from your program.
          \end{itemize}
\end{itemize}

\section{Expected Learning Outcomes}
By completing this assignment, you will:
\begin{itemize}
    \item Understand network resilience and single points of failure.
    \item Implement Tarjan’s Algorithm for articulation point detection.
    \item Simulate and analyze network failures.
\end{itemize}

\end{document}
